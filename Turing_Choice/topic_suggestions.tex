\documentclass{article}
\usepackage[utf8]{inputenc}
\usepackage[T1]{fontenc}
\usepackage{times}

% German
%\usepackage[ngerman]{babel}
% English
\usepackage[english]{babel}

\usepackage[round,authoryear]{natbib}
\usepackage{amsmath,amssymb,amsthm}
\usepackage[hyphens]{url}
\usepackage{graphicx}

\usepackage{lipsum}

\author{Viktor Gsteiger \\ Matriculation Number: 18-054-700}
\title{Topic Suggestions}
\date{20.09.2020 \\\ Seminar: 58826-01 - Turing Award Winners and Their Contributions}

\begin{document}

\maketitle

\section{Preferred Language}
The preferred language of my Seminar topic would be English.

\section{Suggested Topics}
The suggested topics for the Seminar \textit{Turing Award Winners and Their Contributions} are listed in this section in order of preference with the first being the most preferred to the last being the least. However, the preference is only minor and I would be interested in covering every one of them. The year in brackets behind their names is the date on which they received the Turing award. 

\subsection{Peter Naur (2005)}

ALGOL and it's imperative algorithm description is an important pillar of computer science and especially the domain of language design. Most modern programming languages still basically incorporate ALGOL like features. It has been named as one of the most influential early programming languages. ALGOL has been developed by an international consortium of scientists, however, one of the most important contributors to the language is the Danish scientist Peter Naur, who received the Turing award in 2005 for his substantial contribution to ALGOL, compiler design and the practice and art of computer programming. His initiative to found the ALGOL bulletin gave way for a more concentrated effort to establish and design the ALGOL language. The seminar report on Peter Naur would mainly focus on his contribution to the meta aspects of the ALGOL programming language, which, in my opinion, has made it such a great influence on the languages to come. Generally, Peter Naur has always shown to be close to how computer programming was actually done and had great interest in the larger questions rather than academic point of views where for example Edgar Dijkstra was stating how programming should be done. Peter Naur had, with his more liberal approach, a very direct influence on the nowadays popular Agile software development. The report would thus focus on the philosophies of Peter Naur that led to the design of ALGOL and its important influence on the programming languages to come.

\subsection{Kristen Nygaard and Ole-Johan Dahl (2001)}

Object oriented programming has become the main paradigm of computer programming in our times and its effects on the field of computer science has been highly influential. Many aspects of the object oriented programming paradigm could already be found in the simulation language Simula I and the general programming language Simula 67 which have been developed by Kristen Nygaard and Ole-Johan Dahl. By implementation of the important features of object oriented programming languages Nygaard and Dahl provided a framework upon the languages of the future could be built. Especially the importance to logically structure programs as interacting and executing components which later became objects. The seminar paper on the contribution of these two extraordinary men would focus on the important contributions to the field of object oriented programming as well as the dynamic between the two men, both coming from difficult backgrounds, and how it influenced their language design and work in general.

\subsection{Charles William Bachman (1973)}

Data base management systems have become one of the most important corner stones of the field of computer science and have influenced academia and the industry in a strong way. Charles William Bachman has been a pioneer of the data base management systems found in most companies nowadays. With a very diverse career path and a distinguished track record in the field of databases, Charles William Bachman is an unique character. Especially the invention of the Integrated Data Store (IDS) proved to be essential for the large scale data needs of the future. With this first version of a data manipulation language, Charles William Bachman provided application programmers with a strong set of commands to access and manipulate data which has an influence up to this day. My report would especially pay attention to the ingenious IDS, its impact on the industry at the time and its legacy into the 21st century. Furthermore, the influence of the data structure diagrams on network data models could play an important role in the report as this has had a very long lasting influence on the interface between databases and their users.

\section{Conclusion}

I hope that I have presented my favourite topics for this seminar in an interesting way and I have learned a lot already by reading the very general information on those extraordinary gentlemen. Generally, I would be interested in covering all three of them and thus do not really have a strong preference.

\end{document}
