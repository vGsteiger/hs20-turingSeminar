%%%%%%%%%%%%%%%%%%%%%%%%%%%%%%%%%%%%%%%%%
% Beamer Presentation
% LaTeX Template
% Version 1.0 (10/11/12)
%
% This template has been downloaded from:
% http://www.LaTeXTemplates.com
%
% License:
% CC BY-NC-SA 3.0 (http://creativecommons.org/licenses/by-nc-sa/3.0/)
%
%%%%%%%%%%%%%%%%%%%%%%%%%%%%%%%%%%%%%%%%%

%----------------------------------------------------------------------------------------
%	PACKAGES AND THEMES
%----------------------------------------------------------------------------------------

\documentclass{beamer}

\mode<presentation> {

% The Beamer class comes with a number of default slide themes
% which change the colors and layouts of slides. Below this is a list
% of all the themes, uncomment each in turn to see what they look like.

%\usetheme{default}
%\usetheme{AnnArbor}
%\usetheme{Antibes}
%\usetheme{Bergen}
%\usetheme{Berkeley}
%\usetheme{Berlin}
%\usetheme{Boadilla}
%\usetheme{CambridgeUS}
%\usetheme{Copenhagen}
%\usetheme{Darmstadt}
%\usetheme{Dresden}
%\usetheme{Frankfurt}
%\usetheme{Goettingen}
%\usetheme{Hannover}
\usetheme{Ilmenau} % best so far
%\usetheme{JuanLesPins}
%\usetheme{Luebeck}
%\usetheme{Madrid}
%\usetheme{Malmoe}
%\usetheme{Marburg}
%\usetheme{Montpellier}
%\usetheme{PaloAlto}
%\usetheme{Pittsburgh}
%\usetheme{Rochester}
%\usetheme{Singapore}
%\usetheme{Szeged}
%\usetheme{Warsaw}

% As well as themes, the Beamer class has a number of color themes
% for any slide theme. Uncomment each of these in turn to see how it
% changes the colors of your current slide theme.

%\usecolortheme{albatross}
%\usecolortheme{beaver}
%\usecolortheme{beetle}
%\usecolortheme{crane}
%\usecolortheme{dolphin}
%\usecolortheme{dove}
%\usecolortheme{fly}
%\usecolortheme{lily}
\usecolortheme{orchid}
%\usecolortheme{rose}
%\usecolortheme{seagull}
%\usecolortheme{seahorse}
%\usecolortheme{whale}
%\usecolortheme{wolverine}

%\setbeamertemplate{footline} % To remove the footer line in all slides uncomment this line
\setbeamertemplate{footline}[page number] % To replace the footer line in all slides with a simple slide count uncomment this line
%\setbeamertemplate{frametitle continuation}[from second]

%\setbeamertemplate{navigation symbols}{} % To remove the navigation symbols from the bottom of all slides uncomment this line
}

\usepackage{graphicx} % Allows including images
\usepackage{booktabs} % Allows the use of \toprule, \midrule and \bottomrule in tables
\usepackage{csquotes}
\usepackage[font=tiny,labelfont=bf]{caption}
\usepackage{url}
\usepackage{qtree}
\usepackage{float}
\usepackage{tikz}
\usepackage{smartdiagram}
\usepackage{amsmath}
\usepackage{listings}

\usetikzlibrary{automata, positioning, arrows}
\usetikzlibrary{shapes.geometric,calc}

\graphicspath{{figures/}}

\bibliographystyle{plain}

% To quote
\let\oldquote\quote
\let\endoldquote\endquote
\renewenvironment{quote}[2][]
{\if\relax\detokenize{#1}\relax
	\def\quoteauthor{#2}%
	\else
	\def\quoteauthor{#2~---~#1}%
	\fi
	\oldquote}
{\par\nobreak\smallskip\hfill(\quoteauthor)%
	\endoldquote\addvspace{\bigskipamount}}

% To remove header
\makeatletter
\newenvironment{withoutheadline}{
	\setbeamertemplate{headline}[default]
	\def\beamer@entrycode{\vspace*{-\headheight}}
}{}
\makeatother

\makeatletter
\newenvironment<>{proofs}[1][\proofname]{%
	\par
	\def\insertproofname{#1\@addpunct{.}}%
	\usebeamertemplate{proof begin}#2}
{\usebeamertemplate{proof end}}
\makeatother
%\usepackage{natbib}
%\renewcommand\bibfont{\scriptsize}


%----------------------------------------------------------------------------------------
%	TITLE PAGE
%----------------------------------------------------------------------------------------

\title[Formalization]{ALGOL 60 Tutorial} % The short title appears at the bottom of every slide, the full title is only on the title page

\author{Viktor Gsteiger} % Your name
\institute[unibas] % Your institution as it will appear on the bottom of every slide, may be shorthand to save space
{
University of Basel \\ % Your institution for the title page
\medskip
\textit{Seminar Turing Award Winners and Their Contributions} \\
\textit{v.gsteiger@unibas.ch} % Your email address
}
\date{December 17, 2020} % Date, can be changed to a custom date

\begin{document}
\begin{withoutheadline}
\begin{frame}
\titlepage % Print the title page as the first slide
\end{frame}
\end{withoutheadline}

\begin{withoutheadline}
\begin{frame}
\frametitle{Extent of Tutorial}
\begin{itemize}
	\item 35 pages
	\item Bottom up
	\item From Basic to Advanced
	\item Self-Reflective
	\item Example based
\end{itemize}
\end{frame}
\end{withoutheadline}

\begin{withoutheadline}
\begin{frame}[fragile]
	\frametitle{Call-by-name}
	
	\begin{exampleblock}{Example}
		\begin{lstlisting}[language={[60]algol}]
    procedure callByName(arr);
    array arr;
    begin
        integer i;
        i := 0;
        print(arr[i]);
        i := 5;
        print(arr[i]);
        arr[i] := 10
    end;
		\end{lstlisting}
	\end{exampleblock}
\end{frame}
\end{withoutheadline}

\begin{withoutheadline}
\begin{frame}[fragile]
	\frametitle{Call-by-value}
	
	\begin{exampleblock}{Example}
		\begin{lstlisting}[language={[60]algol}]
    integer procedure callByValue(i);
    value i;
    integer i;
    begin
        callByValue := i + 5
    end;
		\end{lstlisting}
	\end{exampleblock}
\end{frame}
\end{withoutheadline}

\begin{withoutheadline}
\begin{frame}[fragile]
	\frametitle{Scope-rules}
	
	\begin{exampleblock}{Example}
		\begin{lstlisting}[language={[60]algol}]
    integer i;
    boolean b;
    b := false;
    i := 0;
    begin
        real r; integer i;
        i := 100;
        r := 5.0;
        begin
            integer b;
            b := 10;
            i := 10
        end;
    end;
		\end{lstlisting}
	\end{exampleblock}
\end{frame}
\end{withoutheadline}

\begin{withoutheadline}
\begin{frame}[fragile]
	\frametitle{While-loops}
	
	\begin{exampleblock}{Example}
		\begin{lstlisting}[language={[60]algol}]
    integer i, max;
    i := 0;
    max := 10;
    WHILE:
        if i >= max then go to EXIT;
        i := i + 1;
        go to WHILE;
    EXIT:
		\end{lstlisting}
	\end{exampleblock}
\end{frame}
\end{withoutheadline}

\begin{withoutheadline}
\begin{frame}[fragile]
	\frametitle{Merge-sort}
	
	\begin{exampleblock}{Example}
	External code
	\end{exampleblock}
\end{frame}
\end{withoutheadline}


\end{document} 