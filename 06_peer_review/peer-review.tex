\documentclass{article}
\usepackage[utf8]{inputenc}
\usepackage[T1]{fontenc}
\usepackage{times}

% German
%\usepackage[ngerman]{babel}
% English
\usepackage[english]{babel}

\usepackage{amsmath,amssymb,amsthm}
\usepackage[hyphens]{url}
\usepackage{graphicx}
\usepackage{hyperref}

\usepackage{lipsum}

% German
%\newtheorem{definition}{Definition}
%\newtheorem{satz}{Satz}
% English
\newtheorem{definition}{Definition}
\newtheorem{theorem}{Theorem}

\title{Peer Review: The impact of how Pearl formalized complex probability models}
\date{25th October 2020}

\begin{document}

\maketitle

\section{Feedback to Individual Sections}


\subsection{Abstract}
The idea of the abstract works and I would generally be interested to know more about Pearls work. I would suggest to rewrite the research fields in a more precise fashion. The currently mentioned fields of influence are very broad and thus hard to grasp. I would also suggest to reformulate the second sentence of the abstract as it is hard to understand the intention of the sentence. I could imagine splitting the sentence up in two sentences to first establish what Pearl made possible to then go on what this contribution allowed other people to do. This could also be done in the abstract in a larger sense. The mixing of contributions and how these contributions influenced others makes it harder to read and makes it seem as though the contributions of Pearl are not connected. The last sentence has some structural problem that the first half does not fit with the second half, I would again suggest to split them up.

\subsection{Introduction}
The introduction gives a rough overview over the topics discussed and is thus quite well in its intention. To explain why the structure the paper is in a chronological order makes sense and establishes an expectation of the reader to also build up their knowledge in order of the topics. I would suggest to stick with this structure, however, I will discuss this in more detail in the \nameref{general}. Concerning the introduction, I would suggest to replace \textit{additionally} with a different connective (and a comma after it) as I assume that the parts on the person behind the accomplishment will be weaved into the other parts of the report as the topics are structured chronologically and thus one could easily build Pearls chronology into this structure.
Lastly, I would suggest to look at sentence structure for this part as there are some words missing and some formulations (e.g. naming) could be replaced by more precise terms (e.g. receiving).

\subsection{Background}
As discussed in the previous part, it makes sense to structure the report in a chronological order. However, I would suggest to rename this section to place it more firmly into the chronological order. Also, think about what this section is trying to accomplish in the grander scheme of the report. Is this section consisting of only biographical information or is the topic of heuristics the main part and the biography the chronological justification how Pearl came to the topic of Heuristics? I would like to read more on the heuristics and especially the part of Pearl to the topic of Heuristics. Therefore, I would not call this section Background but rather something in line with \textit{Pearls Development of Heuristics} with which there is a strong focus on the person as well as the topic of the section.

\subsubsection{Pearl's Influences}
As mentioned in the previous section, it is not clear what the intention of this subsection is if we consider it in the context of the section. The influences of Pearl and the story described in the subsection is interesting and makes a good read, however, it is hard to extrapolate how this influence was important to Pearl. As the whole report is written in chronological order, I would absolutely not remove this part, however, I would focus more on how this particular influence of Pearls teachers have make him work in the way he later on did and how this has led him to the topic of heuristics. This would make a very compelling first argument and would also place importance on the human behind the science as described in the introduction. The title influences however is misleading as there seems to be only one influence described in the section. I would remodel the title in the style of the remodeled subsection and place importance on how this particular influence has lead Pearl to Heuristics or the way he conducted science in general.

\subsubsection{Beginnings}
The beginnings subsection explains in very chronological order more or less the entire scientific career of Pearl by only focusing on his places of work or study. This makes the subsection hard to pin in the section and makes the title misleading as not only the beginning of Pearl is named but also up to where he works now. It is also unclear of what this is supposed to be the beginning. My proposal would be to establish the link between the previous subsection and explain how the high school student Pearl got to the scientific knowledge and insight which allowed him to work in the field of heuristics. This would give the section a common thread and would place it firmly in the idea of the chronological order of scientific development as well as personal development of Pearl. This would also fill the gap between Pearl being a high school student with prolific teachers to him being a scientist himself and would nicely allow to establish the first important field of this young scientist.

\subsubsection{Heuristics}
The subsection on heuristics is lacking a short description and thus it is hard to give a review. I would suggest to take my remarks on the previous subsections into account and weave this subsection into the previous subsections. This has not to be done in this subsection, rather the previous should explain how Pearls development has led him to his works in heuristics. 
It is also important to not that the subsection on heuristics should include the fundamental establishment of the field of work of Pearl as the subsequent work builds upon this. It is also important that the style of notation established in this part must be kept throughout the entire report and that important definitions of terms be made early if used throughout the report.

\subsection{Bayesian Networks}
The title and the description combined with the abstract suggests that this is the most important part of the report. What I would wish to read in a subsection of this section is how Pearl developed these networks, how it ties together with the previous work in heuristics and the previous developments in this field.
Another subsection I would be interested in to read would be the definition of the syntax and how it ties together with the theory of multivariate probabilistic models.
Lastly, I would like to see how this is important and how it has influenced further work in the field and especially the contribution of this syntax to the field of artificial intelligence.
The short description given does capture most points mentioned by me, however, as I have not read any literature in the field of multivariate probabilistic models it is not clear to me what a factor is. It would be of most importance to explain the field in a way, that other students may have an insight without having to read a lot of literature. I know that this is not easy and I have struggled myself with this, however, if you keep to a level of complexity which you yourself would like to read and break down the complex topics in smaller bits it surely can be accomplished.

\subsection{Causality}
The section of causality is explained very briefly, however, the mention of the connection to the topic of Bayesian Networks is important.
In this section I would expect a subsection on how the field of Bayesian Networks and Causality are connected. This would most probably include a part on the problem of Bayesian Networks concerning the projection of casual information.
Furthermore, I would expect a subsection which explains how the Causality developed by Pearl tackles the projection problems of the Bayesian Networks. 
Lastly I would like to read a subsection on how the work of Pearl has influenced further work in this topic and how this is important in the general field of artificial intelligence.

\subsection{Feedback}
\label{feedback}
Thank you for this very honest feedback. I wold suggest you make a very strict and detailed plan for the coming few weeks to be able to finish this report in a timely fashion with all the important parts on board. It could also be possible to leave the topic of Causality out of the report if your time does not allow for it. I would be much more interested in a rather detailed account of one topic than in two broad overviews over two topics. You could then mention the topic of Causality in the further work section and thus not take the importance from it but this would allow you to focus on the most important contribution of Pearl, which, according to your draft, would be the topic of Bayesian Networks.

\subsection{References}
I see some issues with this reference. I would suggest you consult the lecture slides again on how to reference in text. Furthermore, I would suggest you consult the \textit{APA Style} to have a look at how to properly format a video reference. Lastly I would suggest to try to find a reference which is in written form rather than video form, however, this is rather a personal preference.

\section{General feedback}
\label{general}

It is very unfortunate that the report is lacking the large substance of its content, however, I tried to give feedback on the report I received and I hope that my remarks may help the writer to continue their work on this draft structure.
I would suggest to stick to one style of title case capitalization and I would suggest to have a look at the \textit{American Psychological Association Style Guide} as a reference for title capitalization.
It is furthermore important to note that Pearls name has not been consistently capitalized which would make sense for a name.
I would also stress that in light of the upcoming deadline try to focus on one rather than two topics as mentioned in the \nameref{feedback} section. This would allow you to focus your energy and effort into one part of the report and not divide your knowledge up into several parts.
Generally, I would suggest you give your report to someone else before the final hand in to have someone read over the sentence structures and to probably spot some grammatical errors. I will do that as well and it is always advisable to have someone else reading over your work to spot possible errors. Due to the current state of the report this person could not be me but I hope you find someone else to do that before your final hand in.
Concerning the structure, I generally think that the idea of a chronological order makes sense in the context of Pearls work and I would also like to read about the person behind the science through the report to have more context.
I wish you all the best with your report and I am looking forward to the presentation. 
\end{document}
