\documentclass{article}
\usepackage[utf8]{inputenc}
\usepackage[T1]{fontenc}
\usepackage{times}

% German
%\usepackage[ngerman]{babel}
% English
\usepackage[english]{babel}

\usepackage[round,authoryear]{natbib}
\usepackage{amsmath,amssymb,amsthm}
\usepackage[hyphens]{url}
\usepackage{graphicx}

\usepackage{lipsum}

% German
%\newtheorem{definition}{Definition}
%\newtheorem{satz}{Satz}
% English
\newtheorem{definition}{Definition}
\newtheorem{theorem}{Theorem}

% TODO
\author{A. Student}
\title{On Pancakes}
\date{10.09.2020}

\begin{document}

\maketitle

\begin{abstract}
  \lipsum[1]
\end{abstract}

\section{Introduction}
This sentence is justified with a reference \citep{hofstadter-1979}.

This sentence is justified with multiple references \citep{hofstadter-1979,godel-mmp1931,prusinkiewicz-csiro1996}.

\citet{prusinkiewicz-csiro1996} use in-line references.

\lipsum[4]

\section{Background}

\lipsum[1]
\begin{theorem}[Incompleteness theorem, \citeauthor{godel-mmp1931}, \citeyear{godel-mmp1931}]~\\
    If $c$ be a given recursive, consistent class of formulae, then the
    propositional formula which states that $c$ is consistent is not
    $c$-provable; in particular, the consistency of $P$ is unprovable in $P$, it
    being assumed that $P$ is consistent.
\end{theorem}

\subsection{Pancakes}

\lipsum[3]
(See algo in Figure~\ref{fig-pancake}.)


\subsection{Difference to Waffles}
\lipsum[2]


\bibliographystyle{apalike}
\bibliography{references}
\end{document}
